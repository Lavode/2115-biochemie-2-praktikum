\chapter{Procedure}

Procedure is based on the script provided as part of the course, with
adjustments made to certain steps.\cite{skript_ballmoos}

\section{Purification}

\subsection{Memrane solubilization}

\begin{itemize}
	\item Thaw membranes (\SI{10}{\mg\per\ml}) that you prepared on the
		first day.
	\item Create \SI{11}{\ml} dilution according to table
		\ref{tbl:membrane_dilution}
	\item Incubate mixture at \SI{4}{\celsius} for \SI{1}{\hour} while
		stirring slowly.
	\item During solubilisation, prepare buffers used for purification.
	\item Pellet the insolubilized membrane by ultracentrifugation at
		\SI{150000}{G} (Ti60-Rotor: 46000rpm) for \SI{45}{\min} at
		\SI{4}{\celsius}.
	\item Prepare HEPES wash buffer according to table
		\ref{tbl:hepes_buffer_no_ogng}, adjust pH to 7. Split in two
		\SI{250}{\ml} flasks, add \SI{10}{\percent} OGNG to one.
	\item Load \SI{2}{\ml} of Profinity beads on the column.
	\item Once ethanol is removed, add \SI{10}{\ml} (10 column volumes) of
		\SI{50}{\milli\Molar} Sodium acetate pH4,
		\SI{300}{\milli\Molar} \ce{NaCl}.
	\item Load \SI{10}{\ml} of \SI{0.2}{\Molar} Nickel (10 column volumes) 
	\item Wash the column with \SI{10}{\ml} (10 column volumes) of
		\SI{50}{\milli\Molar} Sodium acetate pH4,
		\SI{300}{\milli\Molar} \ce{NaCl}.
	\item Wash the column with \SI{15}{\ml} of MQ water
	\item Equilibrate the resin with \SI{15}{\ml} of wash buffer
		(\SI{10}{\milli\Molar} HEPES pH 7, \SI{200}{\milli\Molar}
		\ce{NaCl}, \SI{5}{\percent} glycerol and \SI{0.1}{\percent}
		OGNG. The Ni-beads are now ready to use.
\end{itemize}

\subsection{Protein purification}

\begin{itemize}
	\item Prepare \SI{40}{\ml} wash buffer with \SI{5}{\milli\Molar}
		Histidine, and \SI{80}{\ml} wash buffer with
		\SI{100}{\milli\Molar} Histidine.
	\item Mix the supernatant with Ni-beads and incubate for \SI{45}{\min}
		at \SI{4}{\celsius} on a rotating wheel.
	\item Load the slurry on the column and collect the flowthrough
		(unbound proteins).
	\item Let flowthrough pass through resin 1 more time in order to bind
		remaining His-tagged protein. Keep an aliquot of \SI{12}{\ul}
		on ice. For cybB561-dsred, keep \SI{30}{\ul}.
	\item Wash the column with \SI{10}{\ml} (10 column volumes) of wash
		buffer containing OGNG. Keep an aliquot of \SI{12}{\ul} on ice.
		For cybB561-dsred, keep \SI{30}{\ul}.
	\item Wash the column with \SI{10}{\ml} (10 column volumes) of wash
		buffer \SI{5}{\milli\Molar} Histidine. Keep an aliquot of
		\SI{12}{\ul} on ice. For cybB561-dsred, keep \SI{30}{\ul}. Why
		do we wash with \SI{5}{\milli\Molar} histidine?
	\item Elute the column with \SI{20}{\ml} wash buffer +
		\SI{100}{\milli\Molar} Histidine. Keep an aliquot of
		\SI{12}{\ul} on ice. For cybB561-dsred, keep \SI{30}{\ul}.
	\item Concentrate the eluate with an Amicon \SI{10}{\kilo\Da} MWCO
		centrifugal concentrator to \SI{0.5}{\ml} (volume needed to
		inject on the size exclusion chromatography column S200
		increase).
\end{itemize}


Only for cybB561-Dsred protein, we will try to separate CybB561 and Dsred in
presence of TEV protease.

\begin{itemize}
	\item take \SI{100}{\ul} of the concentrated CybB561-Dsred and dilute
		it in \SI{10}{\ml} of \SI{10}{\milli\Molar} HEPES pH7,
		\SI{200}{\milli\Molar} \ce{NaCl}, \SI{5}{\percent} glycerol,
		\SI{0.1}{\percent} OGNG.
	\item Add \SI{200}{\ul} of TEV protease (\SI{1}{\mg\per\ml},
		M=\SI{25}{\kilo\Da}, final concentration =
		\SI{0.8}{\micro\Molar})
	\item Incubate overnight at \SI{4}{\celsius} and mix with magnetic
		stirrer.
\end{itemize}



\subsection{Size exclusion chromatography}

After purification of protein of interest using affinity chromatography, you will purify the concentrated protein again using size exclusion chromatography (gel filtration). What is the reason behind this additional step?

\begin{itemize}
	\item S200 column equilibrated overnight using wash buffer with OGNG.
	\item Fill the loop (\SI{0.5}{\ml}) on the ÄKTA pure device with the
		concentrated samples (\SI{0.5}{\ml}).
	\item Start the run. Note that fractions of eluted proteins will be
		collected automatically.
	\item Pool together fractions that you suspect to contain the protein
		of interest.
	\item Concentrate with an Amicon \SI{10}{\kilo\Da} MWCO centrifugal
		concentrator (final volume that can be reached is
		\SI{250}{\ul}).
\end{itemize}

\subsection{Reverse IMAC}

While running a gel filtration, you will do a reverse IMAC with cybB561.Dsred
cleaved overnight using TEV protease as follows:

\begin{itemize}
	\item First, take a sample of \SI{30}{\ul}, keep on ice.
	\item Load the mixture on the column (containing pre-equilibrated resin
		charged again with nickel).
	\item Retain the flow-through. Keep an aliquot of \SI{30}{\ul} on ice.
		Flowthrough was kept as it contained the cleaved HS. The
		flowthrough was light red, the beads not red.
	\item Load elution buffer on the column (the same that you used a day
		before). Keep an aliquot of \SI{30}{\ul} on ice. The elution
		was pink.
\end{itemize}

\subsection{Protein concentration determination}

To determine the concentration of the different purified proteins, we will take
advantage of the absorbance of the 2 hemes that are buried inside the protein.

\begin{itemize}
	\item A buffer \SI{20}{\milli\Molar} HEPES pH7.4, \SI{20}{\milli\Molar}
		\ce{KCl} \SI{200}{\milli\Molar} \ce{NaCl}, \SI{0.05}{\percent}
		DDM was prepared.
	\item A cuvette was filled with \SI{748}{\ul} buffer and the baseline measured by scanning between \SI{600}{\nm} and \SI{400}{\nm}.
	\item \SI{2}{\ul} of protein were added and measured again.
	\item A small amount of sodium dithionite was added to fully reduce the
		protein, and another measurement done.
	\item The original protein concentration was calculated using the
		dilution factor, baseline-compensanted \SI{561}{\nm} absorbance
		difference of oxidized and reduced form, path length of
		\SI{1}{\c} and and extinction coefficient of
		\SI{46.36}{\per\milli\Molar\per\cm}.
	\item Proteins were frozen in liquid nitrogen in small droplets, and
		stored at \SI{-80}{\celsius}.
\end{itemize}

\subsection{SDS-PAGE}

\begin{itemize}
	\item Two gels were prepared according to the recipe (\SI{12}{\percent}).
	\item Samples were prepared according to table \ref{tbl:sds_samples}.
	\item Samples were incubated for \SI{30}{\min} at \SI{40}{\celsius}
	\item \SI{10}{\ul} of sample were loaded per lane, with \SI{5}{\ul} ruler in empty lanes.
	\item The gel was run for \SI{20}{\min} at \SI{125}{\V}, followed by \SI{80}{\min} at \SI{185}{\V}.
	\item The gel was coomassie-stained, and cleaned overnight.
\end{itemize}


\subsection{Fluorescence measurements}

The efficiency of each purification step can be determined by the measurement
of fluorescence of the different samples that were taken in different steps of
purification (this can only be done for cybB561-dsred).

\begin{itemize}
	\item \SI{15}{\ul} of each sample were diluted to \SI{200}{\ul} with
		dd\ce{H2O}.
	\item Diluated samples were loaded onto a 96-well plate.
	\item Fluorescence was measured at an emission of \SI{586}{\nm},
		extinction of \SI{555}{\nm}
\end{itemize}

\begin{table}
	\centering
	\begin{tabu}{llll}
		\toprule
		Component & Stock & Final concentration & Used amount \\
		\midrule
		H158F & \SI{20}{\mg\per\ml} & \SI{10}{\mg\per\ml} & \SI{5.5}{\ml} \\
		OGNG & \SI{9}{\percent} & \SI{1}{\percent} & \SI{12}{\ml} \\
		Glycerol & \SI{85}{\percent} & \SI{5}{\percent} & \SI{0.647}{\ml} \\
		PMSF & \SI{1}{\milli\Molar} & \SI{200}{\milli\Molar} & \SI{55}{\ul} \\
		PEF block & & & some crystals \\
		HEPES Buffer & & & \SI{3.598}{\ml} \\
		\bottomrule
	\end{tabu}
	\caption{\SI{11}{\ml} Membrane dilution}
	\label{tbl:membrane_dilution}
\end{table}

\begin{table}
	\centering
	\begin{tabu}{llll}
		\toprule
		Component & Stock & Final concentration & Used amount \\
		\midrule
		HEPES & \SI{1}{\Molar} & \SI{10}{\milli\Molar} & \SI{5}{\ml} \\
		\ce{NaCl} & \SI{2}{\Molar} & \SI{200}{\milli\Molar} & \SI{50}{\ml} \\
		Glycerol & \SI{85}{\percent} & \SI{5}{\percent} & \SI{29.4}{\ml} \\
		PMSF & \SI{1}{\milli\Molar} & \SI{200}{\milli\Molar} & \SI{55}{\ul} \\
		dd \ce{H2O} & & & \SI{410.05}{\ml} \\
		\bottomrule
	\end{tabu}
	\caption{\SI{50}{\ml} wash buffer}
	\label{tbl:hepes_buffer_no_ogng}
\end{table}

\begin{table}
	\centering
	\begin{tabu}{llllll}
		\toprule
		Gel & Lane & Sample & Protein & Buffer & dd\ce{H2O} \\
		\midrule
		1 & 1 & Ladder & \SI{5}{\ul} & - & - \\
		1 & 2 & wt Flowthrough & \SI{3}{\ul} & \SI{3}{\ul} & \SI{9}{\ul} \\
		1 & 3 & wt Wash & \SI{12}{\ul} & \SI{3}{\ul} & \SI{0}{\ul} \\
		1 & 4 & wt Wash \SI{5}{\milli\Molar} His & \SI{12}{\ul} & \SI{3}{\ul} & \SI{0}{\ul} \\
		1 & 5 & wt before ÄKTA (conc) & \SI{1}{\ul} & \SI{3}{\ul} & \SI{11}{\ul} \\
		1 & 6 & wt after ÄKTA (conc) & \SI{0.5}{\ul} & \SI{3}{\ul} & \SI{11.5}{\ul} \\

		1 & 7 & Ladder & \SI{5}{\ul} & - & - \\
		1 & 8 & mt Flowthrough & \SI{3}{\ul} & \SI{3}{\ul} & \SI{9}{\ul} \\
		1 & 9 & mt Wash & \SI{12}{\ul} & \SI{3}{\ul} & \SI{0}{\ul} \\
		1 & 10 & mt Wash \SI{5}{\milli\Molar} His & \SI{12}{\ul} & \SI{3}{\ul} & \SI{0}{\ul} \\
		1 & 11 & mt before ÄKTA (conc) & \SI{1}{\ul} & \SI{3}{\ul} & \SI{11}{\ul} \\
		1 & 12 & mt after ÄKTA (conc) & \SI{0.5}{\ul} & \SI{3}{\ul} & \SI{11.5}{\ul} \\
		1 & 13 & & & & \\
		1 & 14 & Used by assistant & & & \\


		2 & 1 & Ladder & \SI{5}{\ul} & - & - \\
		2 & 2 & dsRed Flowthrough & \SI{3}{\ul} & \SI{3}{\ul} & \SI{9}{\ul} \\
		2 & 3 & dsRed Wash & \SI{12}{\ul} & \SI{3}{\ul} & \SI{0}{\ul} \\
		2 & 4 & dsRed Wash \SI{5}{\milli\Molar} His & \SI{12}{\ul} & \SI{3}{\ul} & \SI{0}{\ul} \\
		2 & 5 & dsRed before ÄKTA (conc) & \SI{1}{\ul} & \SI{3}{\ul} & \SI{11}{\ul} \\
		2 & 6 & dsRed after ÄKTA (conc) & \SI{0.5}{\ul} & \SI{3}{\ul} & \SI{11.5}{\ul} \\
		2 & 7 & dsRed flowthrough (conc) & \SI{1}{\ul} & \SI{3}{\ul} & \SI{11}{\ul} \\
		2 & 8 & dsRed before ÄKTA (conc) & \SI{1}{\ul} & \SI{3}{\ul} & \SI{11}{\ul} \\
		2 & 9 & dsRed cleaved before loading & \SI{12}{\ul} & \SI{3}{\ul} & \SI{0}{\ul} \\
		2 & 10 & dsRed cleaved flowthrough & \SI{12}{\ul} & \SI{3}{\ul} & \SI{0}{\ul} \\
		2 & 11 & dsRed cleaved elution & \SI{12}{\ul} & \SI{3}{\ul} & \SI{0}{\ul} \\
		2 & 12 & TEV & \SI{2}{\ul} & \SI{3}{\ul} & \SI{10}{\ul} \\
		2 & 13 & wt after ÄKTA & \SI{0.5}{\ul} & \SI{3}{\ul} & \SI{11.5}{\ul} \\
		2 & 14 & dsRed & \SI{8}{\ul} & \SI{3}{\ul} & \SI{4}{\ul} \\
		\bottomrule
	\end{tabu}
	\caption{SDS-PAGE samples}
	\label{tbl:sds_samples}
\end{table}

\section{Enzyme characterization}

After expression and purification, it is time to characterize the different
cybB561 enzymes. From this moment, you will keep your team, but you will work
with all of the enzymes.  Since preparation of liposomes takes time, we
prepared lipids for you as following:  
\begin{itemize}
	\item Put \SI{250}{\ul} of E. coli polar extract (ECPE)
		\SI{20}{\mg\per\ml} dissolved already in chloroform in a
		round-bottom flask. 
	\item Chloroform was evaporated and lipids were dried ON in a
		desiccator.
\end{itemize}

\subsection{Liposomes preparation}

\begin{itemize}
	\item \SI{20}{\milli\Molar} HEPES pH7.4, \SI{20}{\milli\Molar}
		\ce{KCl}, \SI{200}{\milli\Molar} \ce{NaCl} buffer was reused of
		the previous group.
	\item Resuspend dried lipids with \SI{2}{\ml} of buffer (final
		concentration of lipids = \SI{5}{\mg\per\ml}).
	\item Unilamellar liposomes were formed by 7 freeze-thaw cycles in
		liquid nitrogen and at \SI{29.4}{\celsius}, respectively.
	\item Sonicate liposomes on ice in Vibra-Cell VCX sonicator as
		following: 1 cycle of \SI{2.5}{\min} (\SI{30}{\s} on,
		\SI{30}{\s} off) Amplitude \SI{40}{\percent} Keep sonicated
		liposomes on ice till measurements.
\end{itemize}

\subsection{Reconstitution of enzymes into liposomes}

\begin{itemize}
	\item Prepare samples as shown in table \ref{tbl:hs_reconstitution}
	\item Incubate at RT for \SI{30}{\min}
	\item Equilibrate P10 column by washing with \SI{15}{\ml} \SI{10}{\milli\Molar} HEPES buffer
	\item Complete sample to \SI{1}{\ml} with \SI{600}{\ul} buffer
	\item Load sample on column, collect flowthrough.
	\item Elute with \SI{1.5}{\ml} buffer, keep flowthrough.
	\item Wash column with \SI{20}{\ml} dd\ce{H2O}
	\item Ultracentrifuge samples at \SI{200000}{G} for \SI{2}{\hour}. Pellet was red, supernatant clear.
	\item Discard supernatant
	\item Resuspend pellet in \SI{100}{\ul} buffer
\end{itemize}

\section{Activity measurements}

\subsection{Relative enzyme reduction}

\begin{itemize}
	\item Prepare \SI{50}{\ml} of buffer with \SI{20}{\milli\Molar} HEPES,
		\SI{20}{\milli\Molar} \ce{KCl}, \SI{200}{\milli\Molar}
		\ce{NaCl}, \SI{0.1}{\milli\Molar} HXP, \SI{0.1}{\milli\Molar}
		DPTA
	\item Buffers were used as-is for solubilized enzymes, and with \SI{0.05}{\percent} DDM for reconstituted enzymes.
	\item Follow reduction of SOO at \SI{428}{\nm} as follows:
		\begin{itemize}
			\item Add buffer and enzyme
			\item Measure absorption
			\item After \SI{1}{\min} add substrate (XO or quinol, optionally with SOD)
			\item After \SI{3}{\min} add spatula tip DTT
			\item Measure absorption until \SI{4}{\min}
		\end{itemize}
	\item Samples as in table \ref{tbl:relative_reduction_samples} were measured.
\end{itemize}

\subsection{TEV cleavage of Cyb-Dsred reconstituted into liposomes}

\begin{itemize}
	\item Mix:
		\begin{itemize}
			\item \SI{400}{\ul} of reconstituted liposomes 
			\item \SI{200}{\ul} TEV protease (\SI{1}{\mg\per\ml})
			\item \SI{10}{\ml} of buffer \SI{20}{\milli\Molar}
				HEPES pH7.4, \SI{20}{\milli\Molar} \ce{KCl},
				\SI{200}{\milli\Molar} \ce{NaCl}
		\end{itemize}
	\item Incubation at \SI{4}{\celsius} ON on a rotating disc
	\item Same mixture prepared a second time, with \SI{200}{\ul} buffer in
		place of TEV protease.
\end{itemize}

\subsection{Fluorescence measurements and comparison before and after cleavage}

Liposomes were prepared for fluorescence measurements as follows:

\begin{itemize}
	\item Ultracentrifuge the reaction mixture at \SI{100000}{g} for \SI{1}{\hour}
    	\item Discard supernatant (containing Tev protease and cleaved free dsred)
	\item Resuspend the pellet with \SI{400}{\ul} of \SI{20}{\milli\Molar}
		HEPES pH7.4, \SI{20}{\milli\Molar} \ce{KCl},
		\SI{200}{\milli\Molar} \ce{NaCl}
    	\item Store liposomes on ice for further utilization.
	\item Maximum excitation and emission were determined using scan.
		\begin{itemize}
			\item Max excitation: \SI{555}{\nm}
			\item Max emission at \SI{555}{\nm}: \SI{595}{\nm}
		\end{itemize}
	\item Emission at \SI{555}{\nm} measured as follows:
		\begin{itemize}
			\item Buffer and sample added
			\item \SI{1}{\ul} \SI{1}{\Molar} copper solution added
			\item \SI{15}{\ul} \SI{0.5}{\Molar} EDTA added
			\item Emission measured after each step
		\end{itemize}
	\item These three measurements were done for:
		\begin{itemize}
			\item \SI{5}{\ul} solubilized dsRed, \SI{745}{\ul} buffer
			\item \SI{100}{\ul} uncleaved reconstituted dsRed, \SI{650}{\ul} buffer
			\item \SI{100}{\ul} cleaved reconstituted dsRed, \SI{650}{\ul} buffer
			\item \SI{200}{\ul} uncleaved reconstituted dsRed, \SI{550}{\ul} buffer
			\item \SI{200}{\ul} cleaved reconstituted dsRed, \SI{550}{\ul} buffer
		\end{itemize}
\end{itemize}

\subsection{Activity measurements}

In the final part of the experiment the activity of solubilized SOO was
measured in presence of both substrates - superoxide produced from hypoxanthine
by XO, and quinone which was kept from running out by adding BO3 oxidase.

\begin{itemize}
	\item An assay buffer was prepared:
		\begin{itemize}
			\item \SI{50}{\ml} sodium phosphate pH8 \SI{100}{\milli\Molar}
			\item \SI{0.1}{\milli\Molar} DTPA
			\item \SI{0.1}{\milli\Molar} Hypoxanthine
			\item \SI{0.05}{\percent} DDM
			\item \SI{50}{\ul} WST-1 \SI{10}{\milli\Molar}
			\item \SI{50}{\ul} catalase \SI{2}{\mg\per\ml}
		\end{itemize}

	\item Two types of measurement were then performed, each one as follows:
		\begin{itemize}
			\item Added buffer for a final volume of \SI{750}{\ul}
			\item Added varying amount of Q2
			\item Added \SI{60}{\nano\Molar} BO3 oxidase
			\item Measurement of WST-1 reduction at \SI{455}{\nm} was started
			\item After \SI{0.5}{\min} \SI{0.03}{U} XO were added
			\item After \SI{1.5}{\min} a varying amount of SOO was added.
		\end{itemize}

	\item In the first series the quinone concentration was kept constant
		at \SI{100}{\micro\Molar}, and the SOO concentration was varied
		until a concentration was found where SOO was able to quench
		the reduction of WST-1 as well as a control series where SOD
		was added.
	\item For the mutant, measurements were done for the following SOO
		volumes:
		\begin{itemize}
			\item Reference XO, no SOO
			\item Reference XO \& SOD, no SOO
			\item \SI{0.5}{\ul} HS mt
			\item \SI{1}{\ul} HS mt
			\item \SI{2}{\ul} HS mt
			\item \SI{5}{\ul} HS mt
			\item \SI{7}{\ul} HS mt
			\item \SI{9}{\ul} HS mt
		\end{itemize}
	\item And the following volumes for the wild type:
		\begin{itemize}
			\item Reference XO, no SOO
			\item Reference XO \& SOD, no SOO
			\item \SI{0.25}{\ul} HS mt
			\item \SI{0.5}{\ul} HS mt
			\item \SI{1}{\ul} HS mt
		\end{itemize}
	\item In the second series the above determined volumes of SOO of
		\SI{9}{\ul} for the mutant, \SI{1}{\ul} for the wild type were
		kept constant, while the Q2 concentration was varied until well
		within the realm where the reaction was substrate-limited.
	\item The following measurements were done for the mutant:
		\begin{itemize}
			\item \SI{50}{\micro\Molar} Q2
			\item \SI{20}{\micro\Molar} Q2
			\item \SI{10}{\micro\Molar} Q2
			\item \SI{5}{\micro\Molar} Q2
			\item \SI{2}{\micro\Molar} Q2
			\item \SI{0.5}{\micro\Molar} Q2
			\item \SI{0.25}{\micro\Molar} Q2
		\end{itemize}
	\item And for the wild type:
		\begin{itemize}
			\item \SI{25}{\micro\Molar} Q2
			\item \SI{10}{\micro\Molar} Q2
			\item \SI{5}{\micro\Molar} Q2
			\item \SI{1}{\micro\Molar} Q2
			\item \SI{0.2}{\micro\Molar} Q2
			\item \SI{0.1}{\micro\Molar} Q2
			\item \SI{0.05}{\micro\Molar} Q2
			\item \SI{0.01}{\micro\Molar} Q2
		\end{itemize}
\end{itemize}

\begin{table}
	\centering
	\begin{tabu}{llll}
		\toprule
		Sample & Enzyme & Sodium cholate \SI{10}{\percent} & Liposomes \\
		\midrule
		dsRed & \SI{60}{\ul} dsRed & \SI{60}{\ul} & \SI{880}{\ul} \\
		mut & \SI{40}{\ul} mut & \SI{24}{\ul} & \SI{336}{\ul} \\
		wt & \SI{40}{\ul} wt & \SI{24}{\ul} & \SI{336}{\ul} \\
		blank & \SI{150}{\ul} buffer \SI{20}{\milli\Molar} HEPES & & \SI{250}{\ul} \\
		\bottomrule
	\end{tabu}
	\caption{Reconstitution of enzymes into liposomes}
	\label{tbl:hs_reconstitution}
\end{table}

\begin{table}
	\centering
	\begin{tabu}{lllll}
		\toprule
		Measurement & Enzyme & Substrate & Addition & Buffer \\
		\midrule
		1 & \SI{2}{\ul} wt solubilized & \SI{1}{\ul} XO & - & \SI{747}{\ul} \\
		2 & \SI{2}{\ul} wt solubilized & \SI{1}{\ul} XO & \SI{1}{\ul} SOD & \SI{746}{\ul} \\
		3 & \SI{2.2}{\ul} mut solubilized & \SI{1}{\ul} XO & - & \SI{746.8}{\ul} \\
		4 & \SI{2.2}{\ul} mut solubilized & \SI{1}{\ul} XO & \SI{1}{\ul} SOD & \SI{745.8}{\ul} \\
		5 & - (neg control) & \SI{1}{\ul} XO & - & \SI{749.0}{\ul} \\
		6 & \SI{10}{\ul} wt reconstituted & \SI{1}{\ul} XO & - & \SI{739}{\ul} \\
		7 & \SI{10}{\ul} wt reconstituted & \SI{1}{\ul} XO & \SI{1}{\ul} SOD & \SI{738}{\ul} \\
		8 & \SI{10}{\ul} mutt reconstituted & \SI{1}{\ul} XO & - & \SI{739}{\ul} \\
		9 & \SI{10}{\ul} mut reconstituted & \SI{1}{\ul} XO & \SI{1}{\ul} SOD & \SI{738}{\ul} \\
		10 & \SI{10}{\ul} pure liposomes & \SI{1}{\ul} XO & - & \SI{749}{\ul} \\

		11 & \SI{2}{\ul} wt solubilized & \SI{1}{\ul} quinol & - & \SI{747}{\ul} \\
		12 & \SI{2}{\ul} mut solubilized & \SI{1}{\ul} quinol & - & \SI{747}{\ul} \\
		13 & - (neg control) & \SI{1}{\ul} quinol & - & \SI{749}{\ul} \\
		14 & \SI{10}{\ul} wt reconstituted & \SI{1}{\ul} quinol & - & \SI{739}{\ul} \\
		15 & \SI{10}{\ul} mut reconstituted & \SI{1}{\ul} quinol & - & \SI{739}{\ul} \\
		16 & \SI{10}{\ul} pure liposomes & \SI{1}{\ul} quinol & - & \SI{739}{\ul} \\
		\bottomrule
	\end{tabu}
	\caption{Samples for measuring relative reduction}
	\label{tbl:relative_reduction_samples}
\end{table}
