\chapter{Discussion week 1}

\section{Isolation of membranes}

The relatively big difference of roughly \SI{20}{\percent} between the two
usable protein concentrations calculated in section
\ref{sec:protein_concentration}, coupled with two of the measurements being
unusable due to being outside the supported range, is not ideal for determining
the concentration of isolated membrane proteins. The big difference in
concentration hints at one of the dilutions being mixed or pipetted
incorrectly. As only two values are available there is further no information
on which dilution was affected.

However, the concentration of either dilution as well as the average thereof
implies that the protein concentration is high enough to achieve the
\SI{10}{\mg\per\ml} required for further purification steps.

\section{Transformation of bacterial cells}

As the colonies were able to grow on a plate treated with kanamycin we can
conclude that the transformation was successful, and the plasmid - which also
confers resistance to kanamycin - was taken in and had been retained.

\section{Expression screening}

\subsection{Bacteria growth}

All references to data in this section refer to figure
\ref{fig:absorption_expression} and table \ref{tbl:absorption_expression}
respectively.

\subsubsection{IPTG- samples}

Samples which were not treated with IPTG showed the typical growth pattern of
bacterial cells. They started with a lag phase without growth, a log phase with
linear growth, and finally a stationary phase where growth stagnated.
Observation stopped before the decline phase, where cell density would have
lowered, was entered.

\subsubsection{IPTG+ samples}

The IPTG+/Rha- sample behaved as expected too. Shortly after IPTG was added it
induced the overexpression of HS, which caused cell growth to stop as all
energy was put towards expression of HS.

The IPTG+/Rha+ sample behaved mostly like the IPTG- samples, which was
unexpected. As Rhamnose leads to the expression of T7 lysozyme, which inhibits
the T7 polymerase\cite{memstar}, we expected the Rha+/IPTG+ sample to exhibit
growth somewhere in between the IPTG- and the Rha-/IPTG+ samples, as the
inhibition of the T7 polymerase would have limited the overexpression of HS,
allowing the cell to continue growing to some extent.

Possible explanations are that either too much Rhamnose was added, such that
the inhibition of the T7 polymerase was too strong, or that none or too little
IPTG had been added. The subsequent analysis of the protein expression using
fluorescence in section \ref{sec:fluorescence} seems to promote the later.

\subsubsection{Medium comparison}

Looking at how fast cell growth was in the different media it is evident that
cells on LB grew the slowest, while cells on TB and ZYM-505 grew at roughly the
same speed. No conclusion can be reached as far as the final cell density is
concerned, as cells did not reach their stationary phase yet.

\subsection{Protein expression}
\label{sec:fluorescence}

All references to data in this section refer to figure
\ref{fig:expression_fluorescence} and table \ref{tbl:expression_fluorescence}
respectively.

\subsubsection{Rha-/IPTG+}

Comparing the normalized fluorescence measurements of the various samples it is
evident that Rha-/IPTG+ samples had the highest protein expression across all
media. This matches the expectation as HS will be expressed fully, without
anything inhibiting it.

\subsubsection{Cell growth on TB}

Further, cells on the TB media seem to have expressed significantly more HS
than on either the LB or ZYM-505 medium. As fluorescence of the Rha-/IPTG-
sample is significantly higher compared to the Rha-/IPTG- samples of the other
two media, it seems likely that this is due to a higher cell density rather
than an increased number of proteins per cell.

\subsubsection{Protein expression on ZYM-505}

Lastly the ZYM-505 medium increased, without promoting general cell growth
unlike the TB medium, the fluorescence of the Rha-/IPTG+ sample. This implies
that, while the number of cells stayed comparable to the LB medium, the number
of HS proteins proteins per cell has increased.
