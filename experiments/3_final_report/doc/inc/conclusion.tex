\part{Conclusion}

\section{Summary of results}

In this work it was shown that membrane protein yield per call can be increased
by using ZYM-505 auto-induction medium. Further it was shown that superoxide
oxidase can be purified reasonably well using an affinity and size exclusion
chromatography after extraction of membranes.

It was shown that cybB561 is able to be reduced by both superoxide as well as
quinol, confirming that these two are valid substrates. Its $K_m$ with relation
to quinone concentration was determined, with a difference of two to three
orders of magnitude between the wildtype and mutant.

Lastly evidence was found implying that SOO reconstituted into liposomes can
take on both orientations.

\section{Protein expression \& purification}

Being able to achieve a higher yield of membrane proteins allows to
significantly increase the amount of expressed and hence purified protein,
without increasing the requirement for potentially expensive detergent caused
by an increase in number of cells, as would be the result of expression in
TB.\cite{memstar}

Purification results showed that a reasonably pure product could be achieved in
a relatively short amount of time, using readily-available methods. It further
provided information on which steps of purification should be adjusted to
prevent losses, such as ensuring that no protein is lost in the flowthrough of
affinity chromatography.

The ability of rhamnose to slightly inhibit the T7 polymerase, helping to
prevent the buildup of inclusion bodies, could not be observed as expected. A
further attempt of the expression screening part of the experiment should be
made to bring clarity.
\section{Enzyme activity}

CybB is known as an extremely fast, diffusion-limited
enzyme.\cite{superoxide_salvaging}. The difference in $K_m$ values of the H158F
mutation and wildtype implied that the conserved Histidine played a role in the
binding of, or reaction with, substrate.

Additional measurements with lower quinone concentration would allow to more
accurately determine the $K_m$ value.

\section{Membrane protein orientation}

The seemingly non-uniform orientation of HS in liposomes - as opposed to in
vivo where orientation is uniform - is of importance for experiments wanting to
use HS as part of a system modelling parts of the respiratory chain.

A clearer picture could be gathered by repeating cleavage and fluorescence
measurements, taking care to incubate long enough that cleavage would be
complete, as well as adding enough copper to quench fluorescence completely.

