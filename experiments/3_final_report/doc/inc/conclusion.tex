\part{Concluding remarks}

\section{Summary of results}

In this practical it was shown that membrane protein yield per call can be increased
by using ZYM-505 auto-induction medium. Further it was shown that superoxide
oxidase can be purified reasonably well using an affinity and size exclusion
chromatography after extraction of membranes.

It was shown that cybB561 is able to be reduced by both superoxide as well as
quinol, confirming that these two are valid substrates. Its $K_m$ with relation
to quinone concentration was determined, with a difference of two to three
orders of magnitude between \hs{} and \hsmut{}.

Lastly evidence was found implying that SOO reconstituted into liposomes can
take on both orientations.

\section{Protein expression \& purification}

IPTG-induced protein expression in ZYM-505 medium was shown able to achieve a
higher yield of membrane proteins per cell density, compared to LB or TB
medium. This allows increasing the amount of expressed and hence purified
protein without increasing the amount of detergent required, as would be the
case for TB where the total number of cells is higher.\cite{memstar}

Purification results showed that a reasonably pure product could be achieved in
a relatively short amount of time, using readily-available methods. It further
provided information on which steps of purification should be adjusted to
prevent losses, such as ensuring that no protein is lost in the flowthrough of
affinity chromatography.

The ability of Rhamnose to slightly inhibit the T7 polymerase, helping to
prevent the buildup of inclusion bodies and preventing complete stagnation of
cell growth, could not be observed as expected. A further attempt of the
expression screening part of the experiment would have to be performed to bring
clarity.

Further, unlike described in MemStar\cite{memstar} the apparent density of
expressed proteins per cell in TB medium was shown to be higher than in LB or
ZYM-505. Additional protein expression \& concentration determination,
potentially stopped at lower \odbact{} values, would need to be done to be
certain about effective per-cell concentrations in TB.

\section{Enzyme activity}

CybB is known as an extremely fast, diffusion-limited
enzyme.\cite{superoxide_salvaging}. The difference in $K_m$ values of \hsmut{}
and \hs{} which was shown implies that the conserved Histidine at position 158
playes a role in the binding of, or reaction with, quinone as a substrate.

Additional measurements with lower quinone concentration would allow to more
accurately determine the $K_m$ value, as it would offset the impact of single
outliers. Further, rather than relying on a Lineweaver-Burk plot, plotting
relative enzyme activity versus substrate concentration and performing a
nonlinear regression might yield more accurate results.\cite{leskovac2003}

Further experiments could be performed with mutants of eg the
metal-coordinating histidine at position 151, to determine the importance of
this amino acid on enzyme activity. Additionally the Michaelis-Menten constant
of HS in reference to superoxide concentration could be analysed.

\section{Membrane protein orientation}

The seemingly non-uniform orientation of HS in liposomes - as opposed to in
vivo where orientation is uniform - is of importance for experiments wanting to
use reconstituted liposomes with \hs{} as part of a system modelling parts of
the respiratory chain.

A clearer picture could be gathered by repeating cleavage and fluorescence
measurements, taking care to incubate long enough that cleavage would be
complete, as well as adding enough copper to quench fluorescence completely.

