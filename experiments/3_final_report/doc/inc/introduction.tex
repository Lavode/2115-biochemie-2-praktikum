\part{Introduction}

Reactive oxygen species (`ROS') such as superoxide (\ce{O2-}) or hydrogen
peroxide (\ce{H2O2}) are toxic to the body, causing damage to proteins or DNA,
with links to various diseases including cancer. As such organisms under
aerobic conditions counteract ROS via eg Superoxide Dismutase (`SOD') which
catalyzes the partitioning of superoxide into molecular oxygen and hydrogen
peroxide.

One major source of superoxide is the oxidative phosphoryliation within the
mitochondrial matrix. \cite{Novo2008} To a lesser extent these superoxides
diffuse into the cytoplasm, where they are acted upon by SOD. A significant
portion of superoxides however is oxidised by superoxide oxidases - henceafter
referred to as `SOO' or `Halonsaft' due to its colour - a family of
membrane-bound proteins which oxidize superoxides.\cite{superoxide_salvaging}.

One protein of this family is CybB from E. Coli which was shown to function as
a superoxide:ubiquinone oxidoreductase. It oxidizes superoxide to oxygen while
simultaneously reducing quinone to quinol.\cite{superoxide_salvaging} This
enzymatic activity both serves to remove superoxides in close proximity to the
cell membrane, as well as restore the quinone pool, helping to save energy.

% TODO
% - Characterizing (eg activity) we did?
% - Transformation & expression basics
% - Which mutation (H158F)
