\part{Results \& discussion}

\section{Transformation}

The goal of bacterial transformation was to get competent E. Coli cells of the
LEMO21 (DE3) strain to take up pET28a-cybB561 plasmids, allowing for the
selective overexpression of HS in presence of lactose or IPTG. Rhamnose allowed
to selectively inhibit the T7 polymerase translating the HS gene.
\cite{memstar} To permit selection the plasmid further encodes for a resistance
to kanamycin.

The overnight incubated plates containing kanamycin had multiple colonies
growing on them the next day. This showed that the heat shock treatment had
been successful. Transformed colonies were not used further due to time
constraints, rather expression was monitored simultaneously with
pre-transformed starting culture.

\section{Expression}

Note that bacteria resulting from this step were not used further due to time
constraint. Rather purification happened simultaneously, with a provided
bacterial pellet.

\subsection{Expression monitoring}

During expression the efficiency of several media was to be established, both
in terms of speed of cell growth as well as yield of HS per cell. Three
different types of medium (LB, TB, ZYM-505) were prepared, with four samples
each.

Increasing the amount of produced proteins per cell was of interest as it
lowers the amount of detergent required to solublize the cell membranes.

Protein expression was monitored by measurement of \odbact of incubating
samples, visualized in figure \ref{fig:absorption_expression}. A mesurement at
\SI{90}{\min} was excluded as it had been done incorrectly. IPTG was added at
\SI{90}{\min}.

It was shown that cells grew faster in TB and ZYM-505 medium than in LB medium,
as seen by the slope of the curve as well as final \odbact reached.

It was further shown that samples not containing IPTG followed the normal
growth pattern of bacterial cells, starting in a lag phase of little cell
growth, a log phase of exponential cell growth - shown as a lineare incrase of
\odbact - and the beginnings of a stationary phase where cell growth cedes. The
absence of IPTG will have prevented overexpression of HS, allowing cells to
grow as normal.

Cells treated with IPTG at \SI{90}{\min} and not containing Rhamnose showed
stagnation of cell growth indicative of overexpression of HS. Cells treated
with IPTG and containing Rhamnose did not show any sign of overexpression of
HS.

\subsection{Expression efficiency}

After expression the amount of expressed HS-Dsred was measured via fluorescence
measurements. Results were normalized to cell density, giving an approximation
of protein concentration per cell. Such normalized results are are shown in
figure \ref{fig:fluorescence_expression}.

\subsubsection{Base protein expression}

Overall protein expression per cell can be estimated using IPTG-negative
samples. It was shown that LB and ZYM-505 media produce comparable amounts of
proteins per cell, while the amount of proteins per cell in the TB medium was
significantly higher. This conflicts with the memstar paper which found that
protein production normalized to cell density was only slightly higher in TB
compared to the other two media. \cite{memstar}

It is possible that the cell density of the TB samples was underestimated, as
the linear relation between \odbact and cell density only holds for low values
of \odbact, as well as the spectrophotometer being less accurate for values
exceeding $1$.

\subsubsection{HS-Dsred expression}

Comparing the normalized fluorescence of IPTG-positive, Rhamnose-negative
samples with other samples in the same media showed that addition of IPTG
increased the amount of proteins - specifically HS-Dsred - which was expressed
per cell.

\subsubsection{Limiting HS-Dsred expression with Rhamnose}

Addition of Rhamnose to samples was supposed to limit the overexpression of
HS-Dsred, preventing a coplete stagnation of cell growth as well as the
potential buildup of inclusion bodies. \cite{memstar}.

As Rhamnose-positive samples did not show any sign of HS-Dsred expression, with
fluorescence levels comparable to IPTG-negative samples, we conclude that
either too much Rhamnose was added such that the T7 polymerase was inhibited
fully, or that no IPTG had been added to those samples.


\begin{figure}
	\centering
	\begin{subfigure}{\textwidth}
		\includegraphics[width=0.8\linewidth]{../img/absorption_expression.png}
		\caption{\odbact values over time during protein expression}
		\label{fig:absorption_expression}
	\end{subfigure}

	\begin{subfigure}{\textwidth}
		\centering
		\includegraphics[width=0.8\linewidth]{../img/expression_fluorescence.png}
		\caption{Fluorescence of samples after protein expression}
		\label{fig:fluorescence_expression}
	\end{subfigure}

	\caption{Monitoring speed and efficiency of expression}
	\label{fig:expression}
\end{figure}

\section{Purification}

\subsection{Membrane protein concentration}

Concentration of harvested membrane proteins was determined in a BCA assay by
comparison with a known standard. Absorption at \SI{562}{\nm} of the standard
was measured and a linear regression calculated as $\text{absorption} = 0.81542
\cdot \text{concentration} + 0.02701$. Measured absorptions and calculated
concentrations of a dilution of membrae proteins are shown in table
\ref{tbl:bca_absorption_sample}.

Measurements with absorption values outside the standard's range of 0.0053 to
1.6353 were discarded, which resulted in an average protein concentration of
\SI{22.86 \pm 3.20}{\mg\per\ml}.

\begin{table}
	\centering
	\begin{tabu}{llll}
		\toprule
		Absorption & Conc (linear regression) & Dilution factor & Conc (undiluted) \\
		\midrule
		\SI{2.3066}{OD562} & \SI{2.78}{\mg\per\ml} & 1:5  & \SI{13.92}{\mg\per\ml} \\
		\SI{1.8855}{OD562} & \SI{2.28}{\mg\per\ml} & 1:10 & \SI{22.80}{\mg\per\ml} \\
		\SI{0.8671}{OD562} & \SI{1.03}{\mg\per\ml} & 1:20 & \SI{20.60}{\mg\per\ml} \\
		\SI{0.5388}{OD562} & \SI{0.63}{\mg\per\ml} & 1:40 & \SI{25.12}{\mg\per\ml} \\
		\bottomrule
	\end{tabu}
	\caption{OD562 values of sample dilutions}
	\label{tbl:bca_absorption_sample}
\end{table}

\subsection{Purified protein concentration}

Protein concentration was determined using the absorbance of the two hemes at
\SI{561}{\nm} as described. Measurement values between \SIrange{400}{600}{\nm}
are shown in figure \ref{fig:hs_concentration}.

A baseline was established by calculating the average absorption difference of
reduced and oxidized form between \SIrange{580}{600}{\nm}. The absorption
difference at \SI{561}{\nm} was calculated and the baseeline difference added.

Given the path length of the cuvette as \SI{1}{\cm}, a dilution factor of 375
and an extinction coefficient of \SI{46.36}{\per\milli\Molar\per\cm} the
protein concentration was calculated using Beer's law. Such calculated
concentrations are shown in table \ref{tbl:hs_concentration}.

\begin{figure}
	\centering
	\includegraphics[width=0.8\textwidth]{../img/hs_concentration.png}
	\caption{Absorption of HS for concentration determination}
	\label{fig:hs_concentration}
\end{figure}

\begin{table}
	\centering
	\begin{tabu}{lllll}
		\toprule
		Protein & $\Delta_{\text{Absorption at \SI{561}{\nm}}}$ & Baseline & $\Delta_{\text{Absorption at \SI{561}{\nm} corr}}$ & Conc [\si{\milli\Molar}] \\
		\midrule
		Wildtype & 0.042 & 0.004 & 0.046 & 0.372 \\
		Mutant & 0.043 & 0.006 & 0.049 & 0.396 \\
		dsRed & 0.068 & 0.013 & 0.081 & 0.655 \\
		\bottomrule
	\end{tabu}
	\caption{Concentration of purified HS}
	\label{tbl:hs_concentration}
\end{table}

\subsection{SDS PAGE}

Purification steps were checked with an SDS page, results shown in figure
\ref{fig:sds}.

\subsubesction{Wildtype \& mutant}

Cyan marks HS, with an expected molecular weight of
\SI{24}{\kilo\Da}\cite{pdb}. As was expected it was highly concentrated after
the \ce{Ni} bead chromatography, and remained so after the ÄKTA SEC.
Furthermore it can be seen in small quantities in the \SI{5}{\milli\Molar} His
wash implying that a small amount was lost when washing the column. As the band
of the after-ÄKTA steps seems smaller than the one preceding it it is assumed
that some was also lost during the SEC.

\subsubsectoin{Dsred}

Green marks the HS-Dsred pair, with a combined expected weight of
\SI{49}{\kilo\Da} \cite{pdb}. It takes the place of plain HS in the samples of
the wildtype and mutant. After TEV cleavage - that is from the `before loading'
colum on onwards - there is no band at this height anymore, which shows that
cleavage was successful. After cleavage the solo HS is marked in cyan, near its
expected size of \SI{24}{\kilo\Da}. Of interest are faint bands on the same
height in the samples before cleavage, which might imply that a small amount of
HS-Dsred had been cleaved before. In the reverse IMAC it was mostly found in
the flowthrough, as the His tag is on the Dsred. Small amounts can be seen in
the elution, implying that an intermediary wash step might have been required.

Red marks Dsred with its molecular weight of \SI{25}{\kilo\Da}, further
identified by being on the same height as the pure Dsred loaded in the last
lane. It is mostly visible in the elution after cleavage, as its His tag
allowed binding to the column, with tiny amounts visible in the flowthrough. Of
interest once more are the corresponding bands in the lanes before cleavage,
seeming to imply that it undergoes cleavage even in the absence of
TEV-protease.

Dark grey marks the TEV protease with a molecular weight of
\SIrange{25}{27}{\kilo\Da} \cite{pdb}. It was found in the sample taken after
cleavage, as well as the elute of the column due to its His tag. It was not
found in the flowthrough, as its His tag allowed it to bind to the column.

Yellow finally marks what seems to be a dimer of Dsred around
\SI{50}{\kilo\Dalton}. It can be seen in nearly all lanes where Dsred itself is
found.

\begin{figure}
    \centering
    \begin{subfigure}{0.8\textwidth}
        \includegraphics[width=\textwidth]{../img/sds_wt_mut}
        \caption{SDS PAGE gel of wildtype and mutant}
        \label{fig:sds_wt_mut}
    \end{subfigure}

    \begin{subfigure}{0.8\textwidth}
        \includegraphics[width=\textwidth]{../img/sds_dsred_tev_cleavage.png}
        \caption{SDS PAGE gel of dsRed including cleaved dsRed}
        \label{fig:sds_dsred_cleaved}
    \end{subfigure}
    \caption{SDS gels of three protein variants}
    \label{fig:sds}
\end{figure}

\subsection{Purification efficiency}


\section{Characterization}

% Relative reduction, activity
